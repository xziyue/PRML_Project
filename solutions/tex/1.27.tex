\begin{question}{1.27}
	Consider the expected loss for regression problems under the $L_q$ loss function given by
	\begin{align*}
		\ev{L_q} = \int\int \lvert y(\bm{x}) - t \rvert^q p(\bm{x}, t)\ d\bm{x}\ dt.
	\end{align*}
	Write down the condition that $y(\bm{x})$ must satisfy in order to minimize $\ev{L_q}$. Show that, for $q = 1$, this solution represents the conditional median, i.e., the function $y(\bm{x})$ such that the probability mass for $t < y(\bm{x})$ is the same as for $t \geq y(\bm{x})$. Also show that the minimum expected $L_q$ loss for $q \rightarrow 0$ is given by the conditional mode, i.e. by the function $y(\bm{x})$ equal to the value of $t$ that maximizes $p(t \mid \bm{x})$ for each $\bm{x}$.
\end{question}

\begin{answer}{(Wrong! Still Solving!)}
	Our goal is to find expressions of $y(\bm{x})$ in order to minimize $\ev{L_q}$. It can be seen that
	\begin{align}
		\ev{L_q} &= \int\int \lvert y(\bm{x}) - t \rvert^q p(\bm{x}, t)\ dt\ d\bm{x}\\
		&= \int p(\bm{x}) \int \lvert y(\bm{x}) - t \rvert^q p(t \mid \bm{x}) \ dt\ d\bm{x}
	\end{align}
	All quantities in the integral above are nonnegative, and we can try to minimize the integrand, namely
	\begin{align}
		\int \lvert y(\bm{x}) - t \rvert^q p(t \mid \bm{x})\ dt.
	\end{align}
	The optimal $y(\bm{x})$ can be found by setting the partial derivative with respect to itself to be zero. We know that $\lvert a \rvert = \left(x^2 \right)^{1/2}$, therefore we have
	\begin{align}
		&\frac{\partial}{\partial y(\bm{x})}\int \lvert y(\bm{x}) - t \rvert^q p(t \mid \bm{x})\ dt\\
		&= \frac{\partial}{\partial y(\bm{x})}\int \left\{\left[ y(\bm{x}) - t \right]^2\right\}^{\frac{q}{2}} p(t \mid \bm{x})\ dt\\
		&= \int  \frac{q}{2} \left\{\left[ y(\bm{x}) - t \right]^2\right\}^{\frac{q}{2} - 1}\cdot 2\left[ y(\bm{x}) - t \right]p(t \mid \bm{x})\ dt\\
		&= \int q \left\{\left[ y(\bm{x}) - t \right]^2\right\}^{\frac{q-1}{2}} \cdot \frac{ y(\bm{x}) - t }{\left\lvert y(\bm{x}) - t\right\rvert}p(t \mid \bm{x})\ dt\\
		&= q \int \lvert y(\bm{x}) - t \rvert^{q-1} \cdot \sgn{y(\bm{x}) - t }p(t \mid \bm{x})\ dt\\
		&= q \int_{-\infty}^{y(\bm{x})} \left[ y(\bm{x}) - t \right]^{q-1} p(t \mid \bm{x})\ dt - q \int_{y(\bm{x})}^{+\infty} \left[ y(\bm{x}) - t \right]^{q-1}p(t \mid \bm{x})\ dt = 0.
	\end{align}
	Therefore we can see that the stationary condition for $y(\bm{x})$ is
	\begin{gather}
		q \int_{-\infty}^{y(\bm{x})} \left[ y(\bm{x}) - t \right]^{q-1} p(t \mid \bm{x})\ dt - q \int_{y(\bm{x})}^{+\infty} \left[ y(\bm{x}) - t \right]^{q-1}p(t \mid \bm{x})\ dt = 0\\
		 \int_{-\infty}^{y(\bm{x})} \left[ y(\bm{x}) - t \right]^{q-1} p(t \mid \bm{x})\ dt =  \int_{y(\bm{x})}^{+\infty} \left[ y(\bm{x}) - t \right]^{q-1}p(t \mid \bm{x})\ dt.
	\end{gather}
	When $q = 1$, the condition reduces to
	\begin{align}
		\int_{-\infty}^{y(\bm{x})} p(t \mid \bm{x})\ dt =  \int_{y(\bm{x})}^{+\infty}p(t \mid \bm{x})\ dt,
	\end{align}
	which indicates that $y(\bm{x})$ is the conditional median of $t$.
	
	When $q \rightarrow 0$, we need some other method to derive the minimizer. Consider the limit
	\begin{align}
		\lim_{q \rightarrow 0} a^q = 1,
	\end{align}
	where $a \neq 0$. However, the function $\lvert y(\bm{x}) - t \rvert^q$ does not equal to zero unless for some region where $y(\bm{x}) = t$. Suppose that this happens only when $t = t_0$, and on other regions we should always have $\lvert y(\bm{x}) - t \rvert \ > 0$. During the integration, we can exclude a small vicinity around $t_0$, which is denoted by $(t_0 - \tau, t_0 + \tau)$, where $\tau > 0$. Under this setting, it can be seen that
	\begin{align}\label{1.27eqn1}
		&\lim_{q \rightarrow 0} \int_{-\infty}^{+\infty} \lvert y(\bm{x}) - t \rvert^q p(t \mid \bm{x})\ dt\\
		\begin{split}
		&= \lim_{q \rightarrow 0}\int_{-\infty}^{t_0 - \tau} \lvert y(\bm{x}) - t \rvert^q p(t \mid \bm{x})\ dt\\
		&\phantom{=}\quad + \lim_{q \rightarrow 0}\int_{t_0 - \tau}^{t_0 + \tau} \lvert y(\bm{x}) - t \rvert^q p(t \mid \bm{x})\ dt\\
		&\phantom{=}\quad\quad + \lim_{q \rightarrow 0}\int_{t_0 + \tau}^{+\infty} \lvert y(\bm{x}) - t \rvert^q p(t \mid \bm{x})\ dt
		\end{split}\\
		&= \int_{-\infty}^{t_0 - \tau} p(t \mid \bm{x})\ dt + \int_{t_0 + \tau}^{+\infty} p(t \mid \bm{x})\ dt + \lim_{q \rightarrow 0}\int_{t_0 - \tau}^{t_0 + \tau} \lvert y(\bm{x}) - t \rvert^q p(t \mid \bm{x})\ dt\\
		&= 1 - \int_{t_0 - \tau}^{t_0 + \tau} p(t \mid \bm{x})\ dt + \lim_{q \rightarrow 0}\int_{t_0 - \tau}^{t_0 + \tau} \lvert y(\bm{x}) - t \rvert^q p(t \mid \bm{x})\ dt.
	\end{align}
	Assume that the maximum of $\lvert y(\bm{x}) - t \rvert$ on $(t_0 - \tau, t_0 + \tau)$ is $k$. Because $y(\bm{x})$ and $t$ are not correlated in any means, we would assume that $k > 0$ in general. Then we can write
	\begin{gather}
		1 - \int_{t_0 - \tau}^{t_0 + \tau} p(t \mid \bm{x})\ dt + \lim_{q \rightarrow 0}\int_{t_0 - \tau}^{t_0 + \tau} \lvert y(\bm{x}) - t \rvert^q p(t \mid \bm{x})\ dt < 1 - \int_{t_0 - \tau}^{t_0 + \tau} p(t \mid \bm{x})\ dt + \lim_{q \rightarrow 0}\int_{t_0 - \tau}^{t_0 + \tau} k^q p(t \mid \bm{x})\ dt\\
		1 - \int_{t_0 - \tau}^{t_0 + \tau} p(t \mid \bm{x})\ dt + \lim_{q \rightarrow 0}\int_{t_0 - \tau}^{t_0 + \tau} \lvert y(\bm{x}) - t \rvert^q p(t \mid \bm{x})\ dt < 1 - \int_{t_0 - \tau}^{t_0 + \tau} p(t \mid \bm{x})\ dt + \int_{t_0 - \tau}^{t_0 + \tau} p(t \mid \bm{x})\ dt\\
		1 - \int_{t_0 - \tau}^{t_0 + \tau} p(t \mid \bm{x})\ dt + \lim_{q \rightarrow 0}\int_{t_0 - \tau}^{t_0 + \tau} \lvert y(\bm{x}) - t \rvert^q p(t \mid \bm{x})\ dt < 1 - 2\int_{t_0 - \tau}^{t_0 + \tau} p(t \mid \bm{x})\ dt.
	\end{gather}
	This is essentially saying
	\begin{align}
		\lim_{q \rightarrow 0} \int_{-\infty}^{+\infty} \lvert y(\bm{x}) - t \rvert^q p(t \mid \bm{x})\ dt < 1 - 2\int_{t_0 - \tau}^{t_0 + \tau} p(t \mid \bm{x})\ dt.\label{1.27eqn2}
	\end{align}
	We have found an upper bound of (\ref{1.27eqn1}), which almost has nothing to do with the behavior of $y(\bm{x})$ itself, except for where it equals $t$. By using the mean value theorem of definite integrals, it can be seen that there exists $t' \in (t_0 - \tau, t_0 + \tau)$ that satisfies
	\begin{align}
		\int_{t_0 - \tau}^{t_0 + \tau} p(t \mid \bm{x})\ dt = 2\tau p(t' \mid \bm{x}).
	\end{align}
	Combining with (\ref{1.27eqn2}), we can conclude that
	\begin{align}
		\lim_{q \rightarrow 0} \int_{-\infty}^{+\infty} \lvert y(\bm{x}) - t \rvert^q p(t \mid \bm{x})\ dt < 1 - 4\tau p(t' \mid \bm{x}),
	\end{align}
	where $t' \in (t_0 - \tau, t_0 + \tau)$. Because $\tau$ can be arbitrarily small, we can assume that $t' \approx t_0$. In order to minimize (\ref{1.27eqn1}), we would like to choose the $t'$ such that $p(t' \mid \bm{x})$ is the greatest. It is equivalent to choosing a $y(\bm{x})$ such that $p(t_0 \mid \bm{x})$ is the greatest. Therefore we can say that the conditional mode is the minimizer at this point.
\end{answer}