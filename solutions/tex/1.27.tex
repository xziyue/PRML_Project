\begin{question}{1.27}
	Consider the expected loss for regression problems under the $L_q$ loss function given by
	\begin{align*}
		\ev{L_q} = \int\int \lvert y(\bm{x}) - t \rvert^q p(\bm{x}, t)\ d\bm{x}\ dt.
	\end{align*}
	Write down the condition that $y(\bm{x})$ must satisfy in order to minimize $\ev{L_q}$. Show that, for $q = 1$, this solution represents the conditional median, i.e., the function $y(\bm{x})$ such that the probability mass for $t < y(\bm{x})$ is the same as for $t \geq y(\bm{x})$. Also show that the minimum expected $L_q$ loss for $q \rightarrow 0$ is given by the conditional mode, i.e. by the function $y(\bm{x})$ equal to the value of $t$ that maximizes $p(t \mid \bm{x})$ for each $\bm{x}$.
\end{question}

\begin{answer}{}
	Our goal is to find expressions of $y(\bm{x})$ in order to minimize $\ev{L_q}$. It can be seen that
	\begin{align}
		\ev{L_q} &= \int\int \lvert y(\bm{x}) - t \rvert^q p(\bm{x}, t)\ dt\ d\bm{x}\\
		&= \int p(\bm{x}) \int \lvert y(\bm{x}) - t \rvert^q p(t \mid \bm{x}) \ dt\ d\bm{x}
	\end{align}
	All quantities in the integral above are nonnegative, and we can try to minimize the integrand, namely
	\begin{align}
		\int \lvert y(\bm{x}) - t \rvert^q p(t \mid \bm{x})\ dt.
	\end{align}
	The optimal $y(\bm{x})$ can be found by setting the partial derivative with respect to itself to be zero. We know that $\lvert a \rvert = \left(x^2 \right)^{1/2}$, therefore we have
	\begin{align}
		&\frac{\partial}{\partial y(\bm{x})}\int \lvert y(\bm{x}) - t \rvert^q p(t \mid \bm{x})\ dt\\
		&= \frac{\partial}{\partial y(\bm{x})}\int \left\{\left[ y(\bm{x}) - t \right]^2\right\}^{\frac{q}{2}} p(t \mid \bm{x})\ dt\\
		&= \int  \frac{q}{2} \left\{\left[ y(\bm{x}) - t \right]^2\right\}^{\frac{q}{2} - 1}\cdot 2\left[ y(\bm{x}) - t \right]p(t \mid \bm{x})\ dt\\
		&= \int q \left\{\left[ y(\bm{x}) - t \right]^2\right\}^{\frac{q-1}{2}} \cdot \frac{ y(\bm{x}) - t }{\left\lvert y(\bm{x}) - t\right\rvert}p(t \mid \bm{x})\ dt\\
		&= q \int \lvert y(\bm{x}) - t \rvert^{q-1} \cdot \sgn{y(\bm{x}) - t }p(t \mid \bm{x})\ dt\\
		&= q \int_{-\infty}^{y(\bm{x})} \left[ y(\bm{x}) - t \right]^{q-1} p(t \mid \bm{x})\ dt - q \int_{y(\bm{x})}^{+\infty} \left[ y(\bm{x}) - t \right]^{q-1}p(t \mid \bm{x})\ dt = 0.
	\end{align}
	Therefore we can see that the stationary condition for $y(\bm{x})$ is
	\begin{gather}
		q \int_{-\infty}^{y(\bm{x})} \left[ y(\bm{x}) - t \right]^{q-1} p(t \mid \bm{x})\ dt - q \int_{y(\bm{x})}^{+\infty} \left[ y(\bm{x}) - t \right]^{q-1}p(t \mid \bm{x})\ dt = 0\\
		 \int_{-\infty}^{y(\bm{x})} \left[ y(\bm{x}) - t \right]^{q-1} p(t \mid \bm{x})\ dt =  \int_{y(\bm{x})}^{+\infty} \left[ y(\bm{x}) - t \right]^{q-1}p(t \mid \bm{x})\ dt.
	\end{gather}
	When $q = 1$, the condition reduces to
	\begin{align}
		\int_{-\infty}^{y(\bm{x})} p(t \mid \bm{x})\ dt =  \int_{y(\bm{x})}^{+\infty}p(t \mid \bm{x})\ dt,
	\end{align}
	which indicates that $y(\bm{x})$ is the conditional median of $t$.
	
	
\end{answer}