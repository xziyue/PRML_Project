\begin{question}{1.39}
	Consider two binary variables $x$ and $y$ having the joint distribution given in the table as follows:
	\begin{table*}[h]
		\centering
		\begin{tabular}{cc|cc}
			\multicolumn{2}{c}{} & \multicolumn{2}{c}{$y$}\\
			&  & 0 & 1\\ \cline{2-4}
			\multirow{2}{*}{$x$} & 0 & $1/3$ & $1/3$\\
			& 1 & $0$ & $1/3$
		\end{tabular}
	\end{table*}

	\noindent Evaluate the following quantities:
	
	\begin{table*}[h]
		\centering
		\begin{tabular}{l@{\hskip 0.3\linewidth}l@{\hskip 0.3\linewidth}l}
			{\textbf{(a)}$\etpy[x]$} & {\textbf{(c)}$\etpy[y \mid x]$} & {\textbf{(e)}$\etpy[x, y]$}\\ [0.5em]
			
			{\textbf{(b)}$\etpy[y]$} & {\textbf{(d)}$\etpy[x \mid y]$} & {\textbf{(f)}$\mutinfo[x, y]$}
		\end{tabular}
	\end{table*}

	\noindent Draw a diagram to show the relationship between these various quantities.
\end{question}

\begin{answer}{}
	For simplicity, we use the subscript to denote the event of the variable being equal to that value. For example, $p(x_0)$ means $p(x = 0)$. In order to compute these quantities, we need to determine $p(x)$ and $p(y)$, which can be done by marginalizing the joint probability. It can be seen that
	\begin{align}
		p(x_i) &= \sum_{j = 0}^{1} p(x_i, y_j)\\
		p(y_j) &= \sum_{i = 0}^{1} p(x_i, y_j).
	\end{align}
	Therefore we can write
	\begin{align}
		p(x_0) &= 2/3\\
		p(x_1) &= 1/3\\
		p(y_0) &= 1/3\\
		p(y_1) &= 2/3.
	\end{align}
	\begin{enumerate}[label = {\bf (\alph*)}]
		\item \begin{align}
			\etpy[x] = \sum_{i = 0}^{1} p(x_i) \ln p(x_i)
		\end{align}
	\end{enumerate}
\end{answer}