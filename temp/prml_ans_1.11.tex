% Using XeLaTeX for fancy OpenType fonts
\documentclass[10pt]{article}
\usepackage[letterpaper, left=0.75in, right=0.75in, top=1in, bottom=1in]{geometry}
\usepackage[english]{babel}
\usepackage{ragged2e}
\usepackage{fancyhdr}
\usepackage{amsmath}
\usepackage{amssymb}
\usepackage{amsthm}
\usepackage{csquotes}
\usepackage{textcomp}
\usepackage{gensymb}
\usepackage{enumitem}
\usepackage{tcolorbox}
\usepackage{listings}
\usepackage{mathtools}
\usepackage{pifont}
\usepackage{float}
\usepackage{multirow}
\usepackage{comment}
\usepackage{booktabs}
\usepackage{setspace}
\usepackage[boxruled, linesnumbered]{algorithm2e}
\usepackage{caption}
\usepackage{fontspec}
\usepackage{unicode-math}
\usepackage{placeins}
\usepackage{titling}
\usepackage{subcaption}
\usepackage{tikz}
\usepackage{calc}
\usepackage{color}
%\usepackage{hyperref}

\pagestyle{fancy}
\fancyhf{}
\chead{\thetitle}
\cfoot{\thepage}



\newcommand{\prmlqstyle}[1]{{\color{red}\textbf{#1}}}
\newenvironment{question}[1]{\par\medskip\noindent{\prmlqstyle{#1}}~}{\medskip}
\newenvironment{answer}[1]{\par\medskip\noindent\textit{Answer #1.}~}{\medskip}

\renewcommand\qedsymbol{$\blacksquare$}
\newcommand{\bm}[1]{\symbf{#1}}

\newcommand{\T}{\mathsf{T}}

\newcommand{\sgn}[1]{\mathrm{sgn}\left(#1\right)}
\newcommand{\ev}[1]{\mathbb{E}\left[#1\right]}
\newcommand{\var}[1]{\mathrm{var}\left[#1\right]}
\newcommand{\cov}[1]{\mathrm{cov}\left[#1\right]}

\title{Detailed Solutions to PRML}
\author{Ziyue ``Alan'' Xiang}
\date{\today}

\begin{document}
% using STIX2 fonts
\setmainfont {STIX2Text-Regular}[
Extension=.otf,
BoldFont=STIX2Text-Bold,
ItalicFont=STIX2Text-Italic,
BoldItalicFont=STIX2Text-BoldItalic,
]

\setmathfont{STIX2Math.otf}

%\doublespacing
\onehalfspacing

% !! insert begin
\begin{question}{1.11}
	By setting the derivatives of log likelihood function
	\begin{align*}
		\ln p(\bm{x} \mid \mu, \sigma^2) = -\frac{1}{2\sigma^2} \sum_{n=1}^{N} (x_n-\mu)^2 -\frac{N}{2}\ln\sigma^2 - \frac{N}{2}\ln(2\pi),
	\end{align*}
	verify the results of
	\begin{align*}
		\mu_{\mathrm{ML}} &= \frac{1}{N} \sum_{n=1}^{N}x_n,\\
		\sigma^2_{\mathrm{ML}} &= \frac{1}{N} \sum_{n=1}^{N} (x_n - \mu_{\mathrm{ML}})^2.
	\end{align*}
\end{question}

\begin{answer}{}
	By setting the derivative with respect to $\mu$ to zero, we have
	\begin{gather}
		\frac{\partial}{\partial \mu} \left[-\frac{1}{2\sigma^2} \sum_{n=1}^{N} (x_n-\mu)^2 -\frac{N}{2}\ln\sigma^2 - \frac{N}{2}\ln(2\pi)\right] = 0\\
		\frac{\partial}{\partial \mu} \left[-\frac{1}{2\sigma^2} \sum_{n=1}^{N} (x_n-\mu)^2 \right] = 0\\
		\frac{\partial}{\partial \mu} \sum_{n=1}^{N} (x_n^2-2\mu x_n + \mu^2) = 0\\
		\frac{\partial}{\partial \mu} \left[ \sum_{n=1}^{N}x_n^2 - 2\sum_{n=1}^{N}\mu x_n + N\mu^2  \right] = 0\\
		-2\sum_{n=1}^{N}x_n + 2N\mu = 0\\
		\mu = \frac{1}{N} \sum_{n=1}^{N} x_n.
	\end{gather}
	We can conclude that the formula for $\mu_{\mathrm{ML}}$ is correct.
	
	Let $u = \sigma^2$, by setting the derivative with respect to $u$ to zero, we have
	\begin{gather*}
		\frac{\partial}{\partial \sigma^2} \left[-\frac{1}{2\sigma^2} \sum_{n=1}^{N} (x_n-\mu)^2 -\frac{N}{2}\ln\sigma^2 - \frac{N}{2}\ln(2\pi)\right] = 0\\
		\frac{\partial}{\partial u} \left[-\frac{1}{2u} \sum_{n=1}^{N} (x_n-\mu)^2 -\frac{N}{2}\ln u - \frac{N}{2}\ln(2\pi)\right] = 0\\
		\frac{1}{2u^2}  \sum_{n=1}^{N} (x_n-\mu)^2 - \frac{N}{2u} = 0\\
		u = \frac{1}{N} \sum_{n=1}^{N} (x_n-\mu)^2.
	\end{gather*}
	By substituting $\mu$ with $\mu_{\mathrm{ML}}$, we can conclude that the formula for $\sigma^2_{\mathrm{ML}}$ is correct.
\end{answer}
% !! insert end
\end{document}
